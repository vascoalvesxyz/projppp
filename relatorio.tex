% xelatex
\documentclass[a4paper, 12pt]{article}
\usepackage[portuguese]{babel}

\usepackage{fontspec}
\usepackage{microtype}

\usepackage[dvipsnames]{xcolor}
\usepackage{minted}
\usepackage{DejaVuSans}

\usepackage[left=3cm,right=3cm,bottom=3cm]{geometry}

\usepackage{paracol}

\setlength{\parskip}{1ex}%
\setlength{\parindent}{0ex}%

\title{\sffamily\huge\bfseries Relatório\\[1ex]\Large Mini-Projecto PPP}
\author{ } 

\newcommand\code[1]{\texttt{\sloppy #1}}

\begin{document}

\maketitle
\begin{center}
    \large
\begin{tabular}{ r l }
        Vasco Alves &(\texttt{2022228207})\\
        Miguel Silva &(\texttt{2023221244})\\
    \end{tabular}
\end{center}

\tableofcontents
\newpage

\section{Introdução}

\subsection{Objetivo}

O objectivo deste trabalho é construir uma aplicação capaz de armazenar, alterar e gerir a informação dos doentes de um médico hipotético. Esta informação inclui: nome, data de nascimento, cartão de cidadão, telefone e email. Para além desta funcionalidade deve ser capaz de armazenar os seus dados clínicos. Estes são: tensão arterial mínima, tensão arterial máxima, peso e altura. 

\subsection{Funcionalidades }

É necessário um menu através do qual o utilizador interage com todas as funcionalidades do programa. Estas incluem: introduzir e eliminar os dados de um novo doente, listar os doentes por ordem alfabética ou os com que têm tensões máximas acima de um determinado valor, apresentar toda a informação de um doente em especifico; e fazer um registo das tensões, do peso e da altura de um determinado doente num determinado dia.

\section{Implementação do Programa}

Para implementar todas as funcionalidades eficiente decidimos criar duas listas ligadas: uma para armazenar os registos e uma para os doentes. Isto permite gestão de memoria eficiente e ter várias ordenações da lista simultaneamente. As listas ligadas estão armazenadas em ficheiros de texto correspondentes (\code{doentes.txt} e \code{registos.txt}) e são atualizadas sempre que um nó é adicionado ou removido.

\subsection{Estrutura de Ficheiros}

\code{main.c}: é o ficheiro principal onde está declarado o comportamento do programa. Todas as funções e tipos de dados auxiliares estão declaradas em ficheiros externos.  

\code{tipos.h}: todas as estruturas de dados estão declaradas neste ficheiro. Adicionalmente, estão definidas algumas funções para verificar e validar os tipos de dados quando são inseridos pelo utilizador. Os doentes e os registos têm as suas estruturas de dados correspondentes declaradas neste ficheiro. 

\code{tipos.c}: este ficheiro contém a definição das funções que validam os dados inseridos pelo utilizador.


\code{doente.h}: declara todas as funções que manipulam a lista ligada correspondente à informação dos doentes incluindo as funções responsáveis por atualizar e guardar a informação dos doentes em ficheiro.

\code{registo.h}: faz o equivalente a \code{doente.h} para a lista ligada dos registos. 

Por fim, \code{registo.c} e \code{doente.c} contém as definições das funções dos respectivos \textit{header files}.

\subsection{Estruturas de Dados}

\subsection*{\code{> tipos.h}}


\begin{minted}[mathescape]{c}
typedef struct unsigned int size_tt;
\end{minted}
\vspace{-1em}

\begin{paracol}{2}
    \code{size\_tt}: é um \textit{alias} de \code{unsigned int}. É utilizado em outros \textit{header files} quando queremos armazenar um numero inteiro positivo, por exemplo, no id e número de telefone do doente.

    \code{struct Data}: estrutura de dados que guarda o ano, mês e dia correspondentes a uma data. É validada através da função\\ \code{ler\_data}.

    \code{struct Doente}: contém todos os dados respectivos ao doente. As principais funções do programa dependem desta estrutura de dados.

    \code{struct noDoente}: nó da lista ligada respectiva. Têm dois \code{pointers}, um para ordem crescente por id e outro para ordem alfabética.



    \switchcolumn

    \vspace{-1em}
\begin{minted}[mathescape]{c}
typedef struct Data {
  int dia, mes, ano;
} Data;

typedef struct Doente {
  char nome[TAM_EMAIL];
  size_tt id;
  char email[TAM_EMAIL];
  char cc[TAM_CC];
  Data data;       
  size_tt telefone;
} Doente;

typedef struct noDoente {
    struct Doente doente;
    struct noDoente* prox;
    struct noDoente* prox_alfabetica;
} noDoente ;


\end{minted}
\end{paracol}
\vspace{-3em}

\begin{paracol}{2}
    \code{struct Registo}: contém todos os dados respectivos a um registo de um doente. Cada registo contém o id do doente a quem pertence. Pode haver id repetido, o que não é o caso para doentes e permite um doente ter vários registos mas não o contrario. 

    \code{struct noRegisto}: nó da lista ligada respectiva. 



    \switchcolumn

    \vspace{-1em}
\begin{minted}[mathescape]{c}
typedef struct Registo {
  size_tt id;
  Data data; 
  double tensao_minima, tensao_maxima;
  double peso, altura;
} Registo;

typedef struct noRegisto{
  struct Registo registo;
  struct noRegisto* prox;
} noRegisto;
\end{minted}

\end{paracol}


\subsection{Funções}

\subsection*{\code{> tipos.h}}
\begin{minted}[mathescape]{c}
void limpar_buffer(FILE *stream);
void ler_data(Data *data);
int validar_email(char *input);
int validar_nome(char *input);
int validar_cc(char *input);
int validar_telefone(int input);
\end{minted}

\code{limpar\_buffer(stdin)} garante que não existe caracteres que ``sobram'' (como o \code{newline}) depois do utilizador inserir os dados, evitando errors de inserção. As restantes funções são responsáveis por validar os dados associados com \code{struct Doente} quando são pedidos ao utilizador.

\subsection*{\code{> doente.h}}

\begin{minted}[mathescape]{c}
pDoente doente_criar();
void doente_carregar(pDoente raiz);
void doente_guardar(pDoente raiz);
void doente_destroi(pDoente raiz);
int doente_vazia(pDoente raiz);
size_tt doente_obter_id(pDoente raiz);
void doente_insere(pDoente raiz, Doente doente);
void doente_retira(pDoente raiz, size_tt id);
void doente_procura(pDoente raiz, size_tt id, pDoente *anterior, pDoente *atual);
void doente_info(pDoente raiz, size_tt id);
void doente_listar_ordem_alfabetica(pDoente raiz); 
Doente doente_id_para_doente(pDoente raiz, size_tt id);
\end{minted}

\code{doente\_criar}: cria o primeiro nó da lista ligada, a ``raiz'', e retorna o seu endereço.\\
\code{doente\_carregar}: lê os dados guardados em \code{doentes.txt} e insere-os na lista ligada.\\
\code{doente\_guardar}: guarda todos doentes da lista ligada em \code{doentes.txt}.
\code{doente\_destroi}: destroi a lista ligada e liberta toda a memoria que lhe foi alocada.\\
\code{doente\_vazia}: devolve 1 se o endereço para o próximo nó da fila for nulo e 0 se for o contrário. 

\code{doente\_obter\_id}: soma 1 ao maior id da lista ligada evitando que haja ids reutilizados. Parte da funcionalidade do programa afirma que quando as informações de um doente são apagadas o seu id não volta a estar disponível para doentes futuros. O maior id da lista ligada está sempre armazenado no nó raiz. 

\code{doente\_insere}: insere um novo nó na lista ligada na posição certa de acordo com a ordenação por ids e de acordo com a ordem alfabética.

\code{doente\_retira}: retira um doente da lista ligada e liberta a sua memoria.\\
\code{doente\_procura}: procura o endereço do nó anterior por ordem crescente de ids.\\
\code{doente\_procura\_alfabeticamente}: procura o endereço do nó anterior por ordem alfabética. \\
\code{doente\_info}: imprime na consola a informação de o doente correspondente ao id dado como argumento.\\
\code{doente\_listar\_ordem\_alfabetica}: imprime na consola todos os doentes por ordem alfabética.\\
\code{doente\_id\_para\_doente}: devolve um \code{struct Doente} com a informação correspondente ao id pedido.

\subsection*{\code{> registo.h}}

\begin{minted}[mathescape, linenos]{c}
pRegisto registo_criar(); 
void registo_carregar(pRegisto raiz);
void registo_guardar(pRegisto raiz);
void registo_destroi(pRegisto raiz); 
int registo_vazia(pRegisto raiz); 
void registo_insere(pRegisto raiz, Registo registo); 
void registo_retira(pRegisto raiz, size_tt id);

void registo_procura(pRegisto raiz, int chave, pRegisto *anterior, pRegisto *atual);
void registo_procura_id(pRegisto raiz, size_tt id, pRegisto *anterior, pRegisto *atual);
void registo_listar_tensoes_max(pDoente raiz_doente, pRegisto raiz, int n); 
void registo_listar_doente(pRegisto raiz, size_tt id);
int registo_validar_data(Data d_registo, Data d_doente);
\end{minted}

Ás primeiras 7 funções são equivalentes às anteriores mas adaptadas para o \code{struct Registo}.


\code{registo\_procura}:
Procura um registo de saúde com a tensão arterial especificada \code{tensao\_maxima}. Define \code{anterior} para o endereço do nó anterior ao nó encontrado e \code{atual} para o nó encontrado.

\code{registo\_procura\_id}:
Faz o equivalente à função anterior para o ID especificado. 

\code{registo\_listar\_tensoes\_max:} 
Lista todos os registos de saúde com uma tensão arterial (\texttt{tensao\_maxima}) maior ou igual a \texttt{n}, exibindo o nome do paciente e a tensão arterial.

\code{registo\_listar\_doente}:
Lista todos os registos de saúde de um paciente com o ID especificado, exibindo a data, peso, altura e tensão arterial.

\code{registo\_validar\_data}: 
Valida se a data do registo de saúde é igual ou posterior à data especificada do paciente.

%\subsection{\tt main}

\section{Conclusão}
Com este projeto desenvolvemos uma aplicação capaz de gerir eficazmente os dados dos doentes e os seus registos clínicos. Usando listas ligadas organizamos a informação de forma eficiente. A validação dos dados inseridos garante que não haja error desnecessário. Ao guardar os dados em ficheiros de texto garantiu que a informação não é perdida quando a aplicação é encerrada.

Foram aplicados vários conceitos como estruturas de dados, alocação dinâmica de memória e escrita e leitura de ficheiros em C. A divisão do código em múltiplos header files manteve o código organizado e mais fácil de manter.

No futuro, a aplicação pode ser expandida com novas funcionalidades, como por exemplo a integração de uma interface gráfica os dados sensíveis dos doentes.

\end{document}

